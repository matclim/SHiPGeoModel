\documentclass[a4paper,11pt]{article}

\usepackage[margin=2.5cm]{geometry}
\usepackage{hyperref}
\usepackage{listings}
\usepackage{xcolor}
\usepackage{booktabs}
\usepackage{longtable}

\hypersetup{
    colorlinks=true,
    linkcolor=blue,
    urlcolor=blue
}

\lstset{
    basicstyle=\ttfamily\small,
    breaklines=true,
    frame=single,
    backgroundcolor=\color{gray!10}
}

\title{SHiPGeoModel Simulation Documentation}
\author{}
\date{\today}

\begin{document}

\maketitle
\tableofcontents
\newpage

\section{Overview}

\textbf{SHiPGeoModel} is a modular detector geometry and simulation framework built using:

\begin{itemize}
    \item \textbf{GeoModel} for detector geometry description
    \item \textbf{Geant4} for particle transport and physics simulation
\end{itemize}

The framework provides:

\begin{itemize}
    \item Modular calorimeter layer construction
    \item Configurable simulation control
    \item Optional visualization
    \item Support for rotated fibre high-precision layers
    \item Separation of geometry and simulation logic
\end{itemize}

\section{Project Structure}

\begin{lstlisting}
include/        Header files
src/            Implementation files
main_g4.cpp     Geant4 simulation entry point
RunConfig.*     Configuration parsing
geometry.db     Generated GeoModel database
calo.cfg        Example run configuration
\end{lstlisting}

\section{Configuration System}

Simulation parameters are defined in a configuration file (e.g.\ \texttt{run.cfg}).

Example:

\begin{lstlisting}
n_events=100
macro=../run.mac
visualize=1
vis_macro=../vis.mac
\end{lstlisting}

These values are parsed by the \texttt{RunConfig} class.

\section{Core Classes}

\subsection{RunConfig}

\textbf{Files:}
\begin{itemize}
    \item \texttt{include/RunConfig.hh}
    \item \texttt{src/RunConfig.cc}
\end{itemize}

\subsubsection*{Purpose}

Stores runtime configuration parameters parsed from a configuration file.

\subsubsection*{Main Members}

\begin{lstlisting}
int n_events;
std::string macro;
bool visualize;
std::string vis_macro;
\end{lstlisting}

\subsubsection*{Responsibilities}

\begin{itemize}
    \item Parse key-value pairs
    \item Provide typed access to parameters
    \item Control visualization and execution mode
\end{itemize}

\section{main\_g4.cpp}

\subsection*{Purpose}

Entry point of the Geant4 simulation.

\subsection*{Execution Flow}

\begin{enumerate}
    \item Read configuration
    \item Construct \texttt{G4RunManager}
    \item Initialize geometry via \texttt{DetectorConstruction}
    \item Initialize physics list
    \item Optionally start visualization
    \item Execute macro or run events
\end{enumerate}

Visualization is activated if:

\begin{lstlisting}
visualize=1
\end{lstlisting}

\section{DetectorConstruction}

\textbf{Files:}
\begin{itemize}
    \item \texttt{include/DetectorConstruction.hh}
    \item \texttt{src/DetectorConstruction.cc}
\end{itemize}

\subsection*{Purpose}

Defines the full detector geometry for Geant4.

\subsection*{Responsibilities}

\begin{itemize}
    \item Construct world volume
    \item Convert GeoModel geometry to Geant4
    \item Apply visual attributes
    \item Register sensitive detectors
\end{itemize}

\subsection*{Key Method}

\begin{lstlisting}
G4VPhysicalVolume* Construct();
\end{lstlisting}

\section{CalorimeterBuilder}

\textbf{File:}
\begin{itemize}
    \item \texttt{src/CalorimeterBuilder.cpp}
\end{itemize}

\subsection*{Purpose}

Builds the calorimeter stack according to configuration layer codes.

\subsection*{Layer Codes}

\begin{longtable}{ll}
\toprule
Code & Description \\
\midrule
1 & Wide PVT scintillator layer \\
2 & Thin PS scintillator layer \\
5 & Fibre HPLayer (fibres along Y) \\
6 & Fibre HPLayer rotated 90$^\circ$ (fibres along X) \\
7 & Iron absorber \\
8 & Air gap \\
\bottomrule
\end{longtable}

\subsection*{Responsibilities}

\begin{itemize}
    \item Compute total stack thickness
    \item Place layers sequentially along Z
    \item Call appropriate builder classes
\end{itemize}

\section{Fibre\_HPLayer}

\textbf{Files:}
\begin{itemize}
    \item \texttt{include/Fibre\_HPLayer.hh}
    \item \texttt{src/Fibre\_HPLayer.cpp}
\end{itemize}

\subsection*{Purpose}

Constructs high-precision fibre layers inside aluminium casings.

\subsection*{Build Interface}

\begin{lstlisting}
static void build(
    GeoVPhysVol* mother,
    GeoMaterial* aluminumMat,
    GeoMaterial* fiberMat,
    double zCenter_mm,
    int layerIndex,
    double casingXY_mm,
    double casingZ_mm,
    double fiberDiam_mm,
    bool fibresAlongY = true
);
\end{lstlisting}

\subsection*{Orientation Handling}

\begin{itemize}
    \item \texttt{fibresAlongY = true} $\rightarrow$ fibres run along Y
    \item \texttt{fibresAlongY = false} $\rightarrow$ fibres run along X
\end{itemize}

Internally, fibres are constructed as cylinders aligned along Z and rotated:

\begin{itemize}
    \item \texttt{RotateX(90$^\circ$)} converts Z-axis to Y
    \item \texttt{RotateY(-90$^\circ$)} converts Z-axis to X
\end{itemize}

\section{Scintillator Layers}

\subsection{PVTBarLayer}

Constructs wide plastic scintillator bar layers arranged in a grid.

\subsection{FSBarLayer}

Constructs thin sampling scintillator layers.

\section{Materials}

Materials are managed centrally and include:

\begin{itemize}
    \item Aluminium
    \item Polystyrene (fibre material)
    \item Lead
    \item Iron
\end{itemize}

\section{Visualization}

Visualization is controlled via configuration:

\begin{lstlisting}
visualize=1
vis_macro=../vis.mac
\end{lstlisting}

Example macro:

\begin{lstlisting}
/vis/open OGL
/vis/viewer/set/style surface
/vis/drawVolume
/vis/scene/add/axes 0 0 0 1 m
\end{lstlisting}

Visual attributes (color, transparency) are assigned in Geant4 after geometry construction.

\section{Geometry Database}

The GeoModel database can be generated via:

\begin{lstlisting}
./make_leadplate_db
gmex geometry.db
\end{lstlisting}

This allows geometry inspection independent of Geant4 simulation.

\section{Simulation Flow Summary}

\begin{verbatim}
readRunConfigFile()
        ↓
Construct RunManager
        ↓
DetectorConstruction builds geometry
        ↓
CalorimeterBuilder places layers
        ↓
Initialize physics
        ↓
Optional visualization
        ↓
BeamOn / macro execution
\end{verbatim}

\section{Design Philosophy}

\begin{itemize}
    \item Modular geometry components
    \item Configuration-driven simulation
    \item Clear separation between geometry and run control
    \item Support for rotated fibre layers
    \item Extendable layer code system
\end{itemize}

\section{Extending the Simulation}

To add a new layer:

\begin{enumerate}
    \item Define a new layer code
    \item Extend \texttt{CalorimeterBuilder}
    \item Implement a builder class
    \item Update thickness calculations
    \item Optionally define visual attributes
\end{enumerate}

\end{document}
